\chapter{Ranking}

% ---------- @K Metrics ----------
\clearpage
\thispagestyle{rankingstyle}
\section{@K Metrics}
\subsection{@K Metrics}

% ---------- Mean Reciprocal Rank ----------
\clearpage
\thispagestyle{rankingstyle}
\section{Mean Reciprocal Rank}
\subsection{MRR}

The Mean Reciprocal Rank (MRR) is a metric that evaluates the quality of ranked results, typically in information retrieval
and recommendation systems. It measures how high the first relevant result appears in the ranking list. MRR is computed as
the average reciprocal rank across all queries in a dataset.

\begin{center}
    FORMULA GOES HERE
\end{center}

MRR goes from 0 to 1. Where 0 mean no relevant items exist in the ranked results for all queries and 1 is achieved when the
relevant item is always the first result for every query. In practice a low MRR suggests that users may need to scroll
significantly to find the relevant result, leading to poor user experience.

\textbf{When to use MRR?}

Use MRR when the primary goal is to return the most relevant result as close to the top of the list as possible, or
when evaluating the quality of systems where only the position of the first relevant result matters.

\coloredboxes{
    \item Intuitive metric.
    \item Aligns with user expectations in applications where relevance at the top matters most.
}
{
    \item MRR only considers the first relevant result, ignoring subsequent relevant results that might also be important.
    \item Systems returning shorter rankings can sometimes inflate MRR if the first relevant result appears earlier.
    \item MRR doesn't handle well cases where multiple relevant items exist with varying degrees of importance.

}

\clearpage

\thispagestyle{customstyle}

\textbf{Other related metrics}

MRR is a concise and effective metric for assessing ranked output but should be complemented by other metrics like
nDCG (Normalized Discounted Cumulative Gain) or Mean Average Precision when evaluating scenarios with multiple relevant
items or varying relevance grades.


% ---------- Mean Average Precision ----------
\clearpage
\thispagestyle{rankingstyle}
\section{Mean Average Precision}
\subsection{Mean Average Precision}

Mean Average Precision (MAP) is a metric used to evaluate the performance of ranking and recommendation systems. It measures
both the relevance of the recommended items and the order in which they are presented, rewarding systems that rank relevant
items higher. The MAP is calculated by averaging the Average Precision (AP) across all users or queries. 

\begin{center}
    FORMULA GOES HERE
\end{center}

MAP values range from 0 to 1, with 1 indicating a perfect ranking where all relevant items are placed at the top of the list.

\textbf{When to use MAP?}

MAP is particularly useful when evaluating search engines, recommendation systems, and other ranking-based models where
returning relevant items in the top ranks is critical. It is especially suited for situations where multiple queries exist,
and precision at various cutoffs matters more than recall.

\coloredboxes{
    \item Evaluates how well relevant items are ranked.
    \item Handles multiple queries. MAP averages precision over multiple search instances, providing a holistic evaluation
    of the model.
}
{
    \item Sensitive to exact ranking positions. A single misplaced relevant item can significantly impact MAP, making it less
    robust to minor ranking fluctuations.
    \item Requires known relevance labels. MAP relies on predefined relevance judgments, which may not always be
    available or objective.
    \item Complex interpretation.

}

% good resources for inpiration for visual: https://juneandrews.com/2014/12/15/mean-average-precision-isnt-so-nice/
% https://sdsawtelle.github.io/blog/output/mean-average-precision-MAP-for-recommender-systems.html

% ---------- Hit Rate ----------
\clearpage
\thispagestyle{rankingstyle}
\section{Hit Rate}
\subsection{Hit Rate}

Hit Rate is a ranking metric used in recommender systems to measure how often a relevant item appears in the top-k
recommendations. It is a binary metric that checks whether at least one of the user's relevant items is present in the
recommended list. If at least one relevant item appears, it counts as a "hit"; otherwise, it does not.

\begin{center}
    FORMULA GOES HERE
\end{center}

Hit Rate is particularly useful in top-k recommendation scenarios where the primary goal is to ensure that users receive at
least one relevant item in their recommendation lists.

\textbf{When to use Hit Rate?}

Use Hit Rate when evaluating recommender systems that generate ranked lists of recommendations, such as e-commerce product
recommendations, streaming service content suggestions, or news article recommendations. It is best suited for use cases
where delivering at least one relevant recommendation is critical, rather than ranking all relevant items perfectly.

\coloredboxes{
    \item Simple to compute and interpret. Hit Rate provides an intuitive measure of whether the recommender system is
    surfacing at least one relevant item.
    \item Effective for top-k evaluation. This metric is particularly useful when users engage with only a few of
    the top-ranked recommendations.
    \item Works well for sparse data. Since it only requires identifying one correct prediction in the top-k list,
    it is less sensitive to sparsity compared to other ranking metrics.
}
{
    \item Hit Rate does not account for the position of the relevant item within the top-k list. An item in the first
    position has the same weight as an item in the last position.
    \item Hit Rate does not differentiate between multiple relevant hits. This means that if multiple relevant
    items appear in the top-k list, Hit Rate does not increase—it remains the same as long as at least one relevant
    item is found. 

}

% ---------- Normalized Discounted Cumulative Gain ----------
\clearpage
\thispagestyle{rankingstyle}
\section{Normalized Discounted Cumulative Gain}
\subsection{Normalized Discounted Cumulative Gain}

% ---------- Intra-List Diversity ----------
\clearpage
\thispagestyle{rankingstyle}
\section{Intra-List Diversity}
\subsection{Intra-List Diversity}

% ---------- Coverage ----------
\clearpage
\thispagestyle{rankingstyle}
\section{Coverage}
\subsection{Coverage}

% ---------- Behavioral Metrics (Novelty, Serendipity, Diversity) ----------
\clearpage
\thispagestyle{rankingstyle}
\section{Behavioral Metrics}
\subsection{Behavioral Metrics}


