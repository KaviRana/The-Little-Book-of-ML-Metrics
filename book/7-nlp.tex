\chapter{NLP}


% ---------- Bilingual Evaluation Understudy ----------
\clearpage
\thispagestyle{nlpstyle}
\section{BLEU}
\subsection{Bilingual Evaluation Understudy}

BLEU is one of the earliest and most widely used metrics for evaluating machine translation and other
text generation tasks. It measures how similar a machine-generated text is to one or more human
reference translations by calculating the overlap of n-grams (contiguous sequences of words).

% equation
\begin{center}
    FORMULA GOES HERE
\end{center}

BLEU ranges from 0 to 1, where 1 indicates a perfect match with the reference.

\textbf{When to use BLEU?}

Use BLEU when evaluating machine translation, summarization, or text generation models where direct
word or phrase overlap with human-produced references is important.

\coloredboxes{
\item Correlates with human judgments at the corpus level. In its original study, BLEU showed strong
alignment with human evaluators.
\item Supports multiple references. Several human translations can be used, helping BLEU capture
natural variability.
\item To prevent models from producing unrealistically short outputs, BLEU introduces a
brevity penalty.
}
{
\item Insensitive to meaning. BLEU measures surface-level word overlap and does not account for
semantic equivalence.
\item Vulnerable to synonyms and paraphrasing. Valid alternative phrasings are not rewarded.
\item Sensitive to tokenization and normalization choices. BLEU scores can vary depending on how words
are segmented (e.g., handling of punctuation, casing, or subword units).
}

\clearpage

\thispagestyle{customstyle}

\orangebox{Did you know that...}
{BLEU was introduced in 2002 by Papineni et al. at IBM and was the first automatic metric to gain
wide acceptance in machine translation research.}


% ---------- Metric for Evaluation of Translation with Explicit ORdering ----------
\clearpage
\thispagestyle{nlpstyle}
\section{METEOR}
\subsection{Metric for Evaluation of Translation with Explicit ORdering}

METEOR (Metric for Evaluation of Translation with Explicit ORdering) is a machine translation
evaluation metric introduced as an improvement over BLEU. Unlike BLEU, which relies heavily on
exact n-gram matches, METEOR aligns candidate translations with reference translations using stemming,
synonyms, and paraphrases. 

% equation
\begin{center}
    FORMULA GOES HERE
\end{center}

It then computes a harmonic mean of unigram precision and recall, with
recall weighted higher to better capture adequacy. Finally, it introduces a fragmentation penalty to
account for word order, rewarding translations that are both accurate and fluent.

\textbf{When to use METEOR?}

It is especially valuable when translation adequacy (capturing meaning) is as important as fluency.
Because it incorporates linguistic resources such as stemming and synonyms, METEOR is useful for
low-resource or morphologically rich languages, where exact matches are too strict.

\coloredboxes{
\item Closer to human judgment. METEOR was shown to correlate better with human evaluations compared
to BLEU.
\item Flexible matching. Accounts for stemming, synonyms, and paraphrases, making it more tolerant
to variation in expression.
\item Balances adequacy and fluency. By combining precision, recall, and word order penalties,
it reflects both meaning preservation and readability.
}
{
\item Computationally heavy. Requires linguistic resources (stemming, synonym databases),
making it slower than purely statistical metrics.
\item Language dependence. Its quality depends on the availability and completeness of stemming
rules and synonym dictionaries for the target language.
}

\clearpage

\thispagestyle{customstyle}

\orangebox{Did you know that...}
{METEOR was developed at Carnegie Mellon University in 2005 as a direct response to BLEU’s shortcomings.}