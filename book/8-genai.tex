\chapter{GenAI}


% ---------- Perplexity ----------
\clearpage
\thispagestyle{genaistyle}
\section{Perplexity}
\subsection{Perplexity}

% reference:
% https://aclanthology.org/2021.deelio-1.5.pdf
% https://kilthub.cmu.edu/articles/journal_contribution/Evaluation_Metrics_For_Language_Models/6605324/1?file=12095765
% https://huggingface.co/docs/transformers/perplexity
% https://brenocon.com/blog/2013/01/perplexity-as-branching-factor-as-shannon-diversity-index/

Perplexity is a widely used metric for evaluating language models. At its core, perplexity measures how well a probability distribution or model
predicts a sample. Mathematically, it is defined as the exponential of the average negative log-likelihood of the sequence. Intuitively, you can
think of it as the “average branching factor” — the number of equally likely choices the model considers at each step.

% equation
\begin{center}
    FORMULA GOES HERE
\end{center}

A lower perplexity indicates that the model assigns higher probabilities to the observed text, meaning it is less “surprised”.

\textbf{When to use Perplexity?}

Perplexity is most commonly used when evaluating language models during training or benchmarking. It helps track how well a model fits text data and
is especially useful for comparing models trained on the same dataset. However, perplexity is best suited for probabilistic next-token prediction
tasks and less reliable as a direct measure of end-user text quality.

\coloredboxes{
\item Simple and interpretable. Lower perplexity values generally mean better predictive performance.
\item Efficient for training evaluation. It provides a direct signal for model optimization without requiring human evaluation.
}
{
\item Models with low perplexity may still generate incoherent or unhelpful text.
\item Sensitive to tokenization and normalization choices.
\item Not well-defined for masked language models. Architectures like BERT predict missing tokens rather than generating sequences left-to-right,
making perplexity unsuitable for evaluating them.
}

\clearpage

\thispagestyle{customstyle}

\orangebox{Did you know that...}
{Perplexity is closely related to entropy in information theory. In fact, $Perplexity = 2^{H(P)}$
where \(H(P)\) is the entropy. This means that perplexity can be seen as the effective 
number of equally likely words the model is choosing from at each step!}

% ---------- BERTScore ----------
\clearpage
\thispagestyle{genaistyle}
\section{BERTScore}
\subsection{BERTScore}

% reference:
% https://arxiv.org/pdf/1904.09675
% https://wiki.math.uwaterloo.ca/statwiki/index.php?title=BERTScore:_Evaluating_Text_Generation_with_BERT

BERTScore is a metric for evaluating text generation that leverages contextual embeddings from pre-trained language models like BERT.
Instead of relying solely on surface-level n-gram overlap (as in BLEU or ROUGE), BERTScore computes similarity by aligning tokens from the
candidate and reference sentences in embedding space.

% equation
\begin{center}
    FORMULA GOES HERE
\end{center}

Mathematically, for each token in a candidate sentence, BERTScore finds its most similar token in the reference sentence (and vice versa)
using cosine similarity. Precision, recall, and F1 are then aggregated over all pairs, yielding a semantic-oriented score that correlates
strongly with human judgments.

\textbf{When to use the BERTScore?}

Use BERTScore when evaluating tasks where semantic similarity matters more than exact wording, such as machine translation, summarization, or
dialogue generation. It is especially useful when generated text can be phrased differently from references but still convey the same meaning.

\coloredboxes{
\item Context-aware. Uses deep contextual embeddings, capturing meaning beyond surface word matches.
\item Better correlation with humans. Empirical studies show BERTScore aligns more closely with human evaluation than BLEU or ROUGE.
}
{
\item Computationally heavy. Requires embedding extraction with large pre-trained models, making it slower than n-gram metrics.
\item Model dependence. Performance varies depending on which pre-trained model (e.g., BERT, RoBERTa, multilingual-BERT) is used.
\item Bias inheritance. Any biases in the underlying language model embeddings can influence the scores.
}

\clearpage

\thispagestyle{customstyle}

\orangebox{Did you know that...}
{The original BERTScore paper included a picture of Bert from Sesame Street paying homage to the model’s namesake and adding a playful touch to an
otherwise technical paper.}

% ---------- MAUVE ----------
\clearpage
\thispagestyle{genaistyle}
\section{MAUVE}
\subsection{MAUVE}

% reference:
% https://arxiv.org/pdf/2212.14578
% https://arxiv.org/pdf/2102.01454
% https://krishnap25.github.io/mauve/

MAUVE is a metric that measures how close the text written by a model is to the distribution of human text. Instead of checking for word overlap,
it looks at the shape of the two text distributions. It does this by taking samples from both human and model outputs, turning them into embeddings
using a large language model, and then comparing them. The comparison is done using a KL divergence in this embedding space. 

% equation
\begin{center}
    FORMULA GOES HERE
\end{center}

In simple terms, MAUVE asks: “If I look at a pile of human texts and a pile of model texts, do they look like they came from the same source?”
A higher MAUVE score means the model text is statistically closer to human text.

\textbf{When to use the MAUVE?}

Use MAUVE when you want to evaluate open-ended generation tasks like stories, dialogues, or summaries. It’s especially helpful when you care
about both fluency and diversity.

\coloredboxes{
\item Balances quality and diversity. Unlike BLEU or ROUGE, MAUVE doesn’t just check for overlap. It rewards outputs that are fluent and varied.
\item MAUVE has been shown to correlate strongly with how humans rate generated text.
}
{
\item Works best with thousands of examples (the original paper used 5,000). With fewer, results may be unstable.
\item MAUVE only measures similarity as seen by the embedding model. If the style, topics, or length of texts differ, good text can still get
a low score.
\item Heavy to run. By default, it uses GPT-2 large embeddings. Smaller models can be used, but results may differ.
}

\clearpage

\thispagestyle{customstyle}

\orangebox{Did you know that...}
{MAUVE was introduced in 2021 and quickly became popular because it correlated better with human ratings than older metrics like perplexity.
It even won a Outstanding Paper Award at NeurIPS 2021.}

% ---------- Fréchet Inception Distance ---------
\clearpage
\thispagestyle{genaistyle}
\section{FID}
\subsection{Fréchet Inception Distance}

% reference:
% https://papers.nips.cc/paper_files/paper/2017/file/8a1d694707eb0fefe65871369074926d-Paper.pdf
% https://arxiv.org/pdf/2401.09603

The Fréchet Inception Distance (FID) is one of the most widely used metrics to evaluate the quality of images produced by generative models,
such as GANs. Instead of comparing pixels directly, FID compares the distributions of generated and real images in a feature space extracted
from a pretrained Inception-v3 network.

% equation
\begin{center}
    FORMULA GOES HERE
\end{center}

Mathematically, it models these feature distributions as multivariate Gaussians and computes the Fréchet distance Wasserstein-2 distanc between them.
A smaller FID score indicates that generated images are closer to the real ones in terms of both quality and diversity.

\textbf{When to use FID?}

Use FID when you want to evaluate image generation quality and diversity in a way that aligns better with human perception than pixel-level metrics.
It is especially useful in comparing different generative models or monitoring improvements during training.

\coloredboxes{
\item By working in the feature space of a neural network trained on ImageNet, FID captures semantic similarity rather than raw pixel similarity.
\item Diversity-aware. Unlike Inception Score, FID penalizes mode collapse (when the model generates limited varieties of images).
}
{
\item FID is statistically biased. Its expected value on finite samples does not equal the true value, which can lead to misleading
comparisons when datasets are small.
\item Not suited for all setups. FID measures the Wasserstein distance to a fixed ground-truth distribution, so it is inadequate for
evaluating models in domain adaptation or zero-shot generation, where the target distribution is unclear.
\item FID assumes Inception embeddings follow a multivariate normal distribution.
}

\clearpage

\thispagestyle{customstyle}

\orangebox{Did you know that...}
{The Fréchet Inception Distance gets its name from Maurice Fréchet, a French mathematician who introduced the concept of metric spaces in 1906.
More than a century later, his work is now central to evaluating the realism of AI-generated images!}